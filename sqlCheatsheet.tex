\documentclass{sciposter}
\usepackage[cmyk,dvipsnames,usenames,svgnames,table]{xcolor} 
\usepackage{lipsum}
\usepackage{epsfig}
\usepackage{amsmath}
\usepackage{amssymb}
\usepackage[english]{babel}
\usepackage{geometry}
\usepackage{multicol}
\usepackage{graphicx}
\usepackage{tikz}
\usepackage{wrapfig}
\usepackage{gensymb}
\usepackage[utf8]{inputenc}
\usepackage{empheq}
\usepackage[pdfusetitle]{hyperref}
\usepackage{fancyvrb}
\usepackage{xtab}

\usepackage{graphicx,url}
\usepackage{palatino}

\geometry{
 landscape,
 a1paper,
 left=5mm,
 right=50mm,
 top=5mm,
 bottom=50mm,
 }




\providecommand{\tightlist}{%
	\setlength{\itemsep}{0pt}\setlength{\parskip}{0pt}}

%BEGIN LISTINGDEF

\usepackage{listings}
   
\usepackage{inconsolata}

\definecolor{background}{HTML}{fcfeff}
\definecolor{comment}{HTML}{b2979b}
\definecolor{keywords}{HTML}{255957}
\definecolor{basicStyle}{HTML}{6C7680}
\definecolor{variable}{HTML}{001080}
\definecolor{string}{HTML}{c18100}
\definecolor{numbers}{HTML}{3f334d}

\lstdefinestyle{lectureNotesListing}{
	xleftmargin=0.5em, % 2.8 with line numbers
	breaklines=true,
	frame=single,
	framesep=0.6mm,
	frameround=ffff,
	framexleftmargin=0.4em, % 2.4 with line numbers | 0.4 without them
	tabsize=4, %width of tabs
	aboveskip=0.5em,
	classoffset=0,
	sensitive=true,
	% Colors
	backgroundcolor=\color{background},
	basicstyle=\color{basicStyle}\small\ttfamily,
	keywordstyle=\color{keywords},
	commentstyle=\color{comment},
	stringstyle=\color{string},
	numberstyle=\color{numbers},
	identifierstyle=\color{variable},
	showstringspaces=true
}
\lstset{style=lectureNotesListing}

%END LISTINGDEF


%BEGIN TODO DEFINITION
\usepackage{todonotes}
\newcommand{\TODO}[1]{\todo[inline]{\Large TODO:  #1}}
\newcommand*\widefbox[1]{\fbox{\hspace{2em}#1\hspace{2em}}}
%END TODO DEFINITION


%BEGIN OTHER COMMANDS
\renewcommand{\t}[1]{\texttt{#1}}

%BEGIN SETUP
\title{\huge{SQL}}


\author{\Large David Zollikofer}
%END SETUP

% set font
\renewcommand{\familydefault}{\rmdefault}


\begin{document}


\maketitle

\begin{multicols}{3}

This short paper summarizes important SQL concepts. All the content and examples are from \textit{Database Systems: The Complete Book} by Garcia-Molina, Ullman and Widom as well as ETH's lecture slides by Prof.\ Ce Zhang.

\part{SQL for Queries}

\section*{Simple SQL Queries}

A basic SQL query has the form \t{SELECT <attr> FROM <rels> WHERE <cond>} where 
\begin{itemize}
	\item \t{<attr>} are the attributes of the relation we would like to have
	\item \t{<rels>} are the tuples (the relation) we are selecting from
	\item \t{<cond>} filters the tuples in the relation only keeping the ones satisfying the condition
\end{itemize}

A basic query might look as follows:

\begin{lstlisting}[language=SQL]
SELECT title, year, producer FROM
movies WHERE id = 10;
\end{lstlisting}

Conditions are specified with \t{<attr-name> = <value>} and can be chained together using \t{AND, OR, NOT, ()}

\subsection*{String Comparison}

Strings are lexicographically ordered in SQL and can be compared to each other.

We can also use wildcards:
\begin{itemize}
	\item \t{\_} matches exactly one character: \t{WHERE title LIKE 'Ice \_\_\_'} would match any movie that has three characters following the space. (such as Ice age or Ice tea).
	\item\t{\%} matches 0 or more characters: \t{WHERE title LIKE '\%sor\%s'} matches or, order, etc.
\end{itemize}


\subsection*{Dates and Times}
Are prefixed with a keyword \t{DATE, TIMESTAMP, TIME}.
\begin{itemize}
	\item \t{DATE '1999-12-01'} is the first of December of 1999
	\item \t{TIME '15:00:03.25} is 3 and a quarter minutes past three PM.
	\item \t{TIMESTAMP '1966-03-22 13:00:12-1:00'}, notice the \t{-1:00} meaning one hour behind GTM.
\end{itemize}

To extract information you can use \t{SELECT EXTRACT(<selector> FROM <date/time>)} for example \t{SELECT EXTRACT(MONTH FROM '2000-00-12')}


\subsection*{NULL Values and Unknowns}

When a value is not specified in the table it defaults to \t{NULL}.
When we make comparisons and have a \t{NULL} involved the truth value will become an \t{UNKNOWN}. On logical operators in behaves similarly to a \t{FALSE}, though it is not \t{FALSE}. Arithmetic expressions on \t{NULL} result in \t{NULL}.

The database only returns tuples where the \t{WHERE} clause evaluates to true. Hence if we have an \t{UNKNOWN} it will not be returned.


\subsection*{Ordering}

After a \t{WHERE, GROUP BY, HAVING} clause we can have an \t{ORDER BY} clause which looks as follows:

\t{ORDER BY <attrs> <order>}

where \t{<attrs>} is the list of attributes we want to order by and \t{<order>} is either \t{ASC} or \t{DESC} with the default being \t{ASC}.


\subsection*{Union, Intersection and Differences}

Let \t{(Q1)}, \t{(Q2)} be two queries of the form \t{SELECT ... FROM ... WHERE ...}. We now define: 

\begin{itemize}
	\item \t{(Q1) UNION (Q2)} returns records either in \t{(Q1)} or \t{(Q2)} or both.
	\item \t{(Q1) INTERSECT (Q2)} returns only records which are in \t{(Q1)} and \t{(Q2)}
	\item \t{(Q1) EXCEPT (Q2)} returns all records being only in \t{(Q1)} and not in \t{(Q2)}.
\end{itemize}


\subsection*{Case Statements}
We can modify a row dynamically using the case statement. It is especially useful to deal with \t{NULL} values.
\begin{lstlisting}[language=SQL]
SELECT carId, carName,
CASE
	WHEN carAge = 0 THEN 'new'
	WHEN carAge < 20 THEN 'worthless'
	ELSE 'old-timer'
END AS ageComment
FROM cars; 
\end{lstlisting}

\section*{Joins}

\subsection*{Cartesian Product / Cross Join}

\begin{lstlisting}[language=SQL]
SELECT * FROM 
Cars c CROSS JOIN Homeowners h
WHERE c.owner = h.id
\end{lstlisting}

Alternatively there also exist the traditional version with the implicit join:

\begin{lstlisting}[language=SQL]
SELECT * FROM 
Cars c, Homeowners h
WHERE c.owner = h.id
\end{lstlisting}

This query build the cross product between all cards and all homeowners and keeps the records where the car owner identifier is the homeowner id.

\textbf{Note:} Sometimes one also sees \t{Cars as c , Homeowners as h} the keyword \t{as} is completely optional.


When \t{h} and \t{c} both have an attribute \t{id} we have to reference it as either \t{c.id} or \t{h.id} since otherwise there would be ambiguity.

\subsection*{Normal (inner) Join}
Combines tuples from both relations satisfying join condition.
\begin{lstlisting}[language=SQL]
SELECT * FROM 
Cars c JOIN Pets p ON c.owner = p.owner
\end{lstlisting}

or equivalently if both have a column \t{owner} one would use as a join condition:

\begin{lstlisting}[language=SQL]
SELECT * FROM 
Cars c JOIN Pets p USING(owner)
\end{lstlisting}

\subsection*{Natural Joins}
Combines tuples from both relations where all columns with the same name match. In the example if they both had a column \t{owner}, this column would have to match.
\begin{lstlisting}[language=SQL]
SELECT * FROM 
Cars c NATURAL JOIN Pets p
\end{lstlisting}


\subsection*{Outer Joins}

They can be natural or normal. The key difference is that an outer join also includes non matching records and sets the missing attributes to \t{NULL}.
\begin{lstlisting}[language=SQL]
Cars c FULL JOIN Pets p using(owner)
\end{lstlisting}
The statement above combines all pet and car combinations as a normal join but also includes all tuples with no match. (all cars not sharing an owner with a cat and vice versa).

\textbf{Left Outer Join}
\begin{lstlisting}[language=SQL]
Cars c LEFT OUTER JOIN Pets p using(owner)
\end{lstlisting}
The statement above includes all pet and car combinations as a normal join but also includes all cars not sharing an owner with a cat. But the cats not sharing an owner with a car are excluded.

Analogously there exists a \textbf{Right Outer Join}.

\section*{Subqueries}

There are three places where SQL subqueries can occur:

\begin{itemize}
	\item Returning a single value to be used in a \t{WHERE} clause.
	\item Returning relations to be used in a \t{WHERE} clause.
	\item In a \t{FROM} clause followed by a variable to represent that subquery.
\end{itemize}

\subsection*{Single Value Returning}
\begin{lstlisting}[language=SQL]
SELECT * FROM CarOwners co WHERE co.id = (SELECT owner FROM Cars WHERE numberPlate = 123 LIMIT 1)
\end{lstlisting}
Notice that the \t{LIMIT 1} is important. If multiple or no tuples were returned an error would be raised.

\subsection*{Conditions with Relations}

There are four types:

\begin{itemize}
	\item \t{s > ALL R}, is true if s is larger than all values in R.
	\item \t{s > ANY R}, is true if is larger than at least one value in R.
	\item \t{EXISTS R}, is true if R is non-empty.
	\item \t{s IN R}, is true if the tuple s is equal to a tuple in R. (analogously there is \t{NOT IN})
\end{itemize}

As an example: (notice that the parentheses around \t{owner, insurance} are essential to mark it as a tuple). Here we find the car owners who have insurance for their car.
\begin{lstlisting}[language=SQL]
SELECT owner FROM CarOwners, Insurance WHERE (owner, insurance) IN (SELECT owner, insurance FROM OwnerInsurance)
\end{lstlisting}

\subsection*{Correlated Subquery}

We use the tuple we are selecting in a subquery:

\begin{lstlisting}[language=SQL]
SELECT * FROM Cars c
WHERE productionYear <= ALL (
	SELECT year FROM Cars 
	WHERE model = c.model
) 
\end{lstlisting}

The query above find the oldest cards of every model.


\subsection*{Subqueries in \t{FROM} Clause}

\begin{lstlisting}[language=SQL]
SELECT owner FROM
Cars c, (SELECT * FROM OwnerInsurance WHERE payment < 200) cheapInsOwn
WHERE c.owner = cheapInsOwn.owner;
\end{lstlisting}
The query above finds all cars with their respective insurance whose owner has a cheap insurance.

\section*{Full Relation Operations}

\subsection*{Removing Duplicates}

To remove duplicates use a the  \t{DISTINCT} keyword as in \t{SELECT DISTINCT ...}. But note that it is very expensive as all the relations will need to be sorted in order to detect and eliminate duplicates. Note that \t{NULL} will also be included at most once if it occurs in the result.


\subsection*{Recursive Query}
The basic structure is 
\begin{lstlisting}[language=SQL]
WITH RECURSIVE recRel AS(
non_recursive-term
UNION [ALL]
recursive-term
) SELECT * FROM recRel;
\end{lstlisting}
The database executes the \t{recursive-term} as long as the resulting set keeps changing.
\begin{lstlisting}[language=SQL]
WITH RECURSIVE anchestor AS (
	SELECT * FROM people
UNION
	SELECT * from people as p
	WHERE p.father = anchestor.id OR p.mother = anchestor.id
)

SELECT * from anchestor;
\end{lstlisting}

\subsection*{Aggregation Operators}

SQL has five aggregation operators: \texttt{SUM, AVG, MIN, MAX, COUNT}. All except the \t{COUNT} are applied to scalar valued columns. The only excpetion being \t{COUNT(*)} which counts the tuples in the relation constructed from the \t{FROM} and \t{WHERE} clause.

\begin{lstlisting}[language=SQL]
SELECT AVG(netWorth), COUNT(*) 
FROM MovieExec;
\end{lstlisting}


\subsection*{Grouping}
The \t{GROUP BY} clause follows the \t{WHERE} clause and lists arguments that, if two tuples match on these arguments, will place them in the same group. The aggregation operators in the \t{SELECT} are applied on a per group basis. Only attributes mentioned in the \t{GROUP BY} clause may appear in the \t{SELECT} clause.

\begin{lstlisting}[language=SQL]
SELECT studioName, SUM(length)
FROM Movies
GROUP BY studioName;
\end{lstlisting}

\subsection*{Grouping, Aggregation and Nulls}
\begin{itemize}
	\item The value NULL is ignored in aggregation. \t{COUNT(*)} though counts number of tuples, but \t{COUNT(A)} counts number of attributes \t{A} and skips NULLS.
	\item NULL is treated as an ordinary value when forming groups.
	\item Aggregations over empty bag are NULL, except \t{COUNT(*)} being 0.
\end{itemize}


\subsection*{Having}

Assume we would like to filter the groups we have aggregated: We can do this by adding a \t{HAVING} clause. e.g. we only want the names of sellers which have no sale below or equals 500:

\begin{lstlisting}[language=SQL]
SELECT COUNT(*), Name
FROM SellerSale
GROUP BY Name
HAVING MIN(Sale) > 500;
\end{lstlisting}


\section*{Data Modifications}

\subsection*{Insertions}

Basic syntax:

\texttt{
INSERT INTO $R$($A_1,\ldots,A_n$) VALUES ($v_1,\ldots v_n$);
}

if they are already in standard order we can also write:

\texttt{
	INSERT INTO $R$ VALUES ($v_1,\ldots v_n$);
}

or insert programmatically:

\begin{lstlisting}[language=SQL]
INSERT INTO Relation1
	SELECT ....
\end{lstlisting}

We note that values are only inserted after the query has been fully computed preventing interference.

\subsection*{Deletions}
We cannot specify the exact tuple to be deleted, instead we must specify a set of tuples (which can contain 1 tuple) to be deleted.
\begin{lstlisting}[language=SQL]
DELETE FROM R WHERE <cond>
\end{lstlisting}

\subsection*{Updates}
Basic syntax:

\texttt{UPDATE $R$ SET <new vals> WHERE <cond>};

where the new vals are the name of the variable followed by an equals sign and the new assignment. 
\begin{lstlisting}[language=SQL]
UPDATE Professor
SET name = 'Prof. ' || name
WHERE title = 'Professor';
\end{lstlisting}
recall that the \t{||} operator concatenates strings.


\section*{Transactions}


A transaction is a set of SQL commands that get executed atomically. In PostgreSQL every statement in the shell is automatically wrapped in a transaction:

\begin{lstlisting}[language=SQL]
BEGIN;
UPDATE accounts SET balance = balance - 100.00
WHERE name = 'Alice';
SAVEPOINT my_savepoint;
UPDATE accounts SET balance = balance + 100.00
WHERE name = 'Bob';
-- oops ... forget that and use Wally's account
ROLLBACK TO my_savepoint;
UPDATE accounts SET balance = balance + 100.00
WHERE name = 'Wally';
COMMIT;

\end{lstlisting}


\section*{Views}

A view is a virtual relation which is built on a query run on a table. 
\begin{lstlisting}[language=SQL]
CREATE VIEW <viewName> as <viewDefinition>;
\end{lstlisting}
where \t{viewDefinition} is a SQL query.

we can delete a view with \t{DROP VIEW <viewName>;}

\subsection*{Updating Views}

We can only update views satisfying some constraints:
\begin{itemize}
	\item only one base relation involved
	\item the view involves the key of the base relation
	\item view does not involve aggregates, group by or duplicate-elimination
\end{itemize}


\section*{Indexes}

To add an index:

\t{CREATE INDEX <indexName> ON Table1(field1);}

to remove an index:

\t{DROP INDEX <indexName>}

\subsection*{Multicolumn Indexes}

\t{CREATE INDEX <indexName> on Table1(field1,field2);}

Multicolumn indexes first index on field1 and then on field2.





\section*{Isolation Level}

With the comman \texttt{SET TRANSACTION ISOLATION LEVEL <isolation level>} we can set the isolation level for the current transaction.

We can also have a new transaction with a specific isolation level by typing \texttt{BEGIN TRANSACTION ISOLATION LEVEL <isolation level>;}




There are the following levels:

\begin{tabular}{|c|c|c|c|}
	\hline
	& Dirty Reads & Nonrepeatable Reads & Phantoms \\
	\hline
	\texttt{READ UNCOMMITTED}&  &  &  \\
	\hline
	\texttt{READ COMMITTED}& No &  &  \\
	\hline
	\texttt{REPEATABLE READ}& No & No &  \\
	\hline
	\texttt{SERIALIZABLE}& No & No & No \\
	\hline
\end{tabular}


where:

\begin{itemize}
	\item Dirty Reads: When we read data another transaction has modified and not yet committed.
	\item Nonrepeatable Reads: When we read a record twice but the value was allowed to change in between.
	\item Phantoms: When another transactions commits and adds or removes rows to a table we are currently reading. 
\end{itemize}

\vfill\null 
\columnbreak


\part{SQL for Table Management}

\section*{Table Creation}
The basic syntax is as follows:
\begin{lstlisting}[language=SQL]
CREATE TABLE Professor(
	PersNr integer,
	Name varchar(30),
	Level character(2) default "AP",
	PRIMARY KEY (PersNr)
);
\end{lstlisting}
The default is a value that will be used if no \t{Level} is specified (otherwise we would have a \t{NULL}).

\subsection*{Data Types}

\begin{itemize}
	\item \t{int, bit} integer, 0-1 bit
	\item \t{numeric(precision)} a number with fixed precision
	\item \t{boolean} 
	\item \t{timestamp, date, time}
	\item \t{varchar(n)} variable length string with limit
	\item \t{char(n)} string which will be padded to size n
	\item \t{text} unlimited string (in theory)
\end{itemize}

\section*{Relations}

\subsection*{Declaring Foreign Keys}

We have two options:
\begin{lstlisting}[language=SQL]
CREATE TABLE Dog(
	name VARCHAR(20),
	owner INT REFERENCES Person(id)
);
\end{lstlisting}

\begin{lstlisting}[language=SQL]
CREATE TABLE Dog(
	name VARCHAR(20),
	owner INT,
	FOREIGN KEY (owner) REFERENCES Person(id)
);
\end{lstlisting}

With the second version we can also define composite foreign keys by writing

\t{FOREIGN KEY (<attributes>) REFERENCES <table>(<attributes>)}

\subsection*{Referential Integrity}

What should happen to a record (called child record in the following discussion) if a parent record (the one referred to in one of the childs foreign key fields) gets updated or deleted?

There are multiple options: (usually the key a child refers to is a parents primary key)

\begin{itemize}
	\item \t{CASCADE}
	\begin{itemize} 
		\item \t{UPDATE} if the parents primary key is updated the child's foreign key will also be updated.
		\item \t{DELETE} if parent gets deleted, child will be deleted too.
	\end{itemize}
	\item \t{RESTRICT}
	\begin{itemize} 
		\item \t{UPDATE} attempt to update parent's primary key will fail with an error
		\item \t{DELETE} attempt to update parent's primary key will fail with an error
	\end{itemize}
	\item \t{NO ACTION}
	\begin{itemize} 
		\item \t{UPDATE} attempt to update parent's primary key will fail with an error
		\item \t{DELETE} attempt to update parent's primary key will fail with an error
	\end{itemize}
	\item \t{SET DEFAULT}
	\begin{itemize} 
		\item \t{UPDATE} when parent's primary key gets updated child's foreign key will be set to DEFAULT or NULL
		\item \t{DELETE} when parent gets deleted child's foreign key will be set to DEFAULT or NULL
	\end{itemize}
\end{itemize}

\textbf{Difference between \t{NO ACTION} and \t{RESTRICT}?} With \t{RESTRICT} the constraint will be checked and immediately fails whereas with \t{NO ACTION} the database tries to update/delete but then fails on committing (failure of deferred check). However some databases (e.g. MySQL) do not differentiate between the two options.

Usually \t{NO ACTION} is the \textbf{default} if no specific constraint has been chosen.

The constraints can be specified during table creation as follows:

\begin{lstlisting}[language=SQL]
create table Lecture (
	...,
	PersNr integer references Professor
	on delete {cascade | restrict | no action |
		set null |set default}
	on update {cascade | restrict | no action |
		set null |set default}
); 
\end{lstlisting}

\subsection*{Enforcing 1:1 Relations}

This is best done using a third table:
\begin{lstlisting}[language=SQL]
CREATE TABLE Employee(
	emp_id VARCHAR(20) NOT NULL UNIQUE
);
CREATE TABLE Office(
	room VARCHAR(20) NOT NULL UNIQUE
);

CREATE TABLE HasOffice(
	emp_id VARCHAR(20) NOT NULL UNIQUE REFERENCES Employee (emp_id),
	room VARCHAR(20) NOT NULL UNIQUE REFERENCES Office (room)
);
\end{lstlisting}

\section*{Constraints}

\subsection*{\t{NOT NULL}}
Attribute cannot be \t{NULL}
\begin{lstlisting}[language=SQL]
..
sName text NOT NULL,
...

\end{lstlisting}
\subsection*{\t{PRIMARY KEY}}
Only one primary key constraint can be specified per relation.
Two options, either a single attribute as primary key like
\begin{lstlisting}[language=SQL]
...
sID int PRIMARY KEY,
...
\end{lstlisting}
or a combination of attributes:
\begin{lstlisting}[language=SQL]
CREATE TABLE Student
(
	sID int,
	sName text,
	GPA real
	PRIMARY KEY (sID, sName)
);
\end{lstlisting}
where the combination must be unique and all attributes need to be non null.
\subsection*{\t{UNIQUE}}
Usage like primary key: A value must be unique, but different from \t{PRIMARY KEY} the value can be \t{NULL} and we can have multiple \t{UNIQUE} constraints per relation.

\subsection*{\t{CHECK}}
Can happen at tuple or attribute level which will get checked at update of tuple or single attribute respectively.
\begin{lstlisting}[language=SQL]
CREATE TABLE Employee
(
salary real,
bonus real,
CHECK(salary + bonus > 10000)
);
\end{lstlisting}

\begin{lstlisting}[language=SQL]
CREATE TABLE Employee
(
salary real CHECK(salary  > 5000)
);
\end{lstlisting}


\section*{Assertions \& Triggers}

\subsection*{Assertion}
A logical expression that must be true at all times:
\begin{lstlisting}[language=SQL]
CREATE  ASSERTION <assertName> CHECK (<condition>)
\end{lstlisting}

\subsection*{Trigger}
A trigger is an SQL statement that gets executed when a certain predefined action is performed. It is best to just look at an example or read the docs if a trigger is needed:

\begin{lstlisting}[language=SQL]
CREATE TRIGGER CeoCantGetPayCut
AFTER UPDATE OF (salary) ON Ceo
REFERENCING
	OLD ROW AS oldRow,
	NEW ROW AS newRow,
FOR EACH ROW
WHEN(oldRow.salary > newRow.salary)
	UPDATE Ceo
	SET salary = oldRow.salary
	WHERE id = oldRow.id;
\end{lstlisting}



\newpage


\end{multicols}
\end{document}